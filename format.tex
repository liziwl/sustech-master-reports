% !Mode:: "TeX:UTF-8"
\setlength{\subfigbottomskip}{0pt}
\CTEXsetup[name={,},number={}]{chapter}
\captionsetup{labelsep=space,font=small,justification=centering}
\arraycolsep=1.7pt
\graphicspath{{figures/}}
\renewcommand{\subcapsize}{\zihao{5}}
\renewcommand{\thesubfigure}{\alph{subfigure})}
\setcounter{secnumdepth}{4}
\newcommand{\pozhehao}{\raisebox{0.1em}{------}}
\titleformat{\chapter}{\center\zihao{-2}\heiti}{\chaptertitlename}{0.5em}{}
\titlespacing{\chapter}{0pt}{-4.5mm}{8mm}
\titleformat{\section}{\zihao{-3}\heiti}{\thesection}{0.5em}{}
\titlespacing{\section}{0pt}{4.5mm}{4.5mm}
\titleformat{\subsection}{\zihao{4}\heiti}{\thesubsection}{0.5em}{}
\titlespacing{\subsection}{0pt}{4mm}{4mm}
\titleformat{\subsubsection}{\zihao{-4}\heiti}{\thesubsubsection}{0.5em}{}
\titlespacing{\subsubsection}{0pt}{0pt}{0pt}
\makeatletter
\renewcommand\thesection{\@arabic \c@section} % 前面不带 thechapter
\makeatother

\newif\ifxueweidoctor
\newif\ifxueweimaster
\def\temp{Doctor}
\ifx\temp\xuewei
  \xueweidoctortrue  \xueweimasterfalse
\fi
\def\temp{Master}
\ifx\temp\xuewei
  \xueweidoctorfalse  \xueweimastertrue
\fi

\def\temp{English}
\ifx\temp\preferlanguage
  % English
  \renewcommand{\contentsname}{Table of Contents}
  \renewcommand{\figurename}{Fig.}
  \renewcommand{\tablename}{Table}
  \renewcommand\thesubfigure{(\alph{subfigure})}
  \CTEXoptions[bibname={Reference}]
  \newtheorem{definition}{Definition}
  \newtheorem{example}{Example}
  \newtheorem{algo}{Algorithm}
  \newtheorem{theorem}{Theorem}
  \newtheorem{axiom}{Axiom}
  \newtheorem{proposition}{Proposition}
  \newtheorem{lemma}{Lemma}
  \newtheorem{corollary}{Corollary}
  \newtheorem{remark}{Remark}
  \renewcommand{\proofname}{\bf Proof}
  \renewcommand{\qedsymbol}{$\blacksquare$} % 证毕符号改成黑色的正方形
\else
  % Chinese
  \CTEXoptions[bibname={主要参考文献}]
  \def\equationautorefname{式}%
  \def\footnoteautorefname{脚注}%
  \def\itemautorefname{项}%
  \def\figureautorefname{图}%
  \def\tableautorefname{表}%
  \def\partautorefname{篇}%
  \def\appendixautorefname{附录}%
  \def\chapterautorefname{章}%
  \def\sectionautorefname{节}%
  \def\subsectionautorefname{小小节}%
  \def\paragraphautorefname{段落}%
  \def\subparagraphautorefname{子段落}%
  \def\FancyVerbLineautorefname{行}%
  \def\theoremautorefname{定理}%
  \newtheorem{definition}{\indent \heiti 定义}
  \newtheorem{example}{\indent \heiti 例}
  \newtheorem{algo}{\indent \heiti 算法}
  \newtheorem{theorem}{\indent \heiti 定理}
  \newtheorem{axiom}{\indent \heiti 公理}
  \newtheorem{proposition}{\indent \heiti 命题}
  \newtheorem{lemma}{\indent \heiti 引理}
  \newtheorem{corollary}{\indent \heiti 推论}
  \newtheorem{remark}{\indent \heiti 注解}
  \renewcommand{\proofname}{\indent \heiti 证明}
  \renewcommand{\qedsymbol}{$\blacksquare$} % 证毕符号改成黑色的正方形
\fi

% 定义页眉和页脚 使用fancyhdr 宏包
% \newcommand{\makeheadrule}{
% \rule[7pt]{\textwidth}{0.75pt} \\[-23pt]
% \rule{\textwidth}{2.25pt}}
% \renewcommand{\headrule}{
%     {\if@fancyplain\let\headrulewidth\plainheadrulewidth\fi
%      \makeheadrule}}
% \makeatother

\pagestyle{fancyplain}
\renewcommand{\chaptermark}[1]{\relax}
\renewcommand{\sectionmark}[1]{\markright{#1}}
\fancyhf{}
\renewcommand{\headrulewidth}{0pt}
\ifxueweidoctor
  % \fancyhead[CO]{\songti \zihao{-5}\rightmark}
  % \fancyhead[CE]{\songti \zihao{-5} 南方科技大学博士学位论文开题报告}%
  \fancyfoot[C]{\zihao{-5} -~\thepage~-}
	% \renewcommand\bibsection{\section*{\centerline{\bibname}}
	% \markboth{南方科技大学博士学位论文开题报告}{\bibname}}
\fi
\ifxueweimaster
  % \fancyhead[C]{\songti \zihao{-5} 南方科技大学硕士学位论文开题报告}
  \fancyfoot[C]{\zihao{-5} -~\thepage~-}
	% \renewcommand\bibsection{\section*{\centerline{\bibname}}
	% \markboth{南方科技大学硕士学位论文开题报告}{\bibname}}
\fi

\renewcommand{\CJKglue}{\hskip 0.56pt plus 0.08\baselineskip} %加大字间距,使每行33个字
\def\defaultfont{\renewcommand{\baselinestretch}{1.62}\normalsize\selectfont}
% 调整罗列环境的布局
\setitemize{leftmargin=3em,itemsep=0em,partopsep=0em,parsep=0em,topsep=-0em}
\setenumerate{leftmargin=3em,itemsep=0em,partopsep=0em,parsep=0em,topsep=0em}
\renewcommand{\theequation}{\arabic{equation}}
\renewcommand{\thetable}{\arabic{table}}
\renewcommand{\thefigure}{\arabic{figure}}

\makeatletter
\renewcommand{\p@subfigure}{\thefigure~}
\makeatother

\newcommand{\citeup}[1]{\textsuperscript{\cite{#1}}} % for WinEdt users

% 封面、摘要、版权、致谢格式定义
\makeatletter
\def\title#1{\def\@title{#1}}\def\@title{}
\def\titlesec#1{\def\@titlesec{& \rule[-4pt]{200pt}{1pt}\hspace{-326pt}\centerline{\textbf{#1}}}}\def\@titlesec{}
\def\affil#1{\def\@affil{#1}}\def\@affil{}
\def\subject#1{\def\@subject{#1}}\def\@subject{}
\def\author#1{\def\@author{#1}}\def\@author{}
\def\bdate#1{\def\@bdate{#1}}\def\@bdate{}
\def\supervisor#1{\def\@supervisor{#1}}\def\@supervisor{}
\def\assosupervisor#1{\def\@assosupervisor{\textbf{副\hfill 导\hfill 师} & \rule[-4pt]{200pt}{1pt}\hspace{-326pt}\centerline{\textbf {#1}}\\}}\def\@assosupervisor{}
\def\cosupervisor#1{\def\@cosupervisor{\textbf{联\hfill 合\hfill 导\hfill 师} & \rule[-4pt]{200pt}{1pt}\hspace{-326pt}\centerline{\textbf {#1}}\\}}\def\@cosupervisor{}
\def\date#1{\def\@date{#1}}\def\@date{}
\def\stuno#1{\def\@stuno{#1}}\def\@stuno{}
% 定义封面
\ifxueweidoctor
\def\makecover{
    \thispagestyle{empty}
    \zihao{-2}\vspace*{10mm}
		\renewcommand{\CJKglue}{\hskip 2pt plus 0.08\baselineskip}
    \centerline{\kaishu\textbf{南方科技大学}}
		\vspace{10mm}
		\centerline{\zihao{2}\songti\textbf{博士学位论文开题报告}}
		\renewcommand{\CJKglue}{\hskip 0pt plus 0.08\baselineskip}

    \zihao{3}\vspace{2\baselineskip}
    \hspace*{36pt}{\songti
	\renewcommand{\arraystretch}{1.3}
    \begin{tabular}{l@{}l}
    \textbf{院\hfill (系)}   & \rule[-4pt]{200pt}{1pt}\hspace{-326pt}\centerline{\textbf\@affil}\\
    \textbf{学\hfill 科}     & \rule[-4pt]{200pt}{1pt}\hspace{-326pt}\centerline{\textbf\@subject}\\
    \textbf{导\hfill 师}     & \rule[-4pt]{200pt}{1pt}\hspace{-326pt}\centerline{\textbf\@supervisor}\\
    \@assosupervisor
	\@cosupervisor
    \textbf{研\hfill 究\hfill 生}      & \rule[-4pt]{200pt}{1pt}\hspace{-326pt}\centerline{\textbf\@author}\\
    \textbf{入\hfill 学\hfill 时\hfill 间}  & \rule[-4pt]{200pt}{1pt}\hspace{-326pt}\centerline{\textbf\@bdate}\\
    \textbf{开题报告日期} & \rule[-4pt]{200pt}{1pt}\hspace{-326pt}\centerline{\textbf\@date}\\
    \textbf{论\hfill 文\hfill 题\hfill 目}  & \rule[-4pt]{200pt}{1pt}\hspace{-326pt}\centerline{\textbf\@title}\\
    \@titlesec
    \end{tabular}\renewcommand{\arraystretch}{1}}
	\vfill
    \centerline{\songti\textbf{研究生院}}
}
\fi

\def\temp{Proposal}
\ifx\temp\baogao
\def\report@name@en{Master's Thesis Proposal}
\def\report@name{硕士学位论文开题报告}
\def\date@name@en{Date of Proposal Report}
\def\date@name{开题报告日期}
\else
\def\report@name@en{Study Progress Report of Master's Students}
\def\report@name{硕士研究生年度考核报告}
\def\date@name@en{Submission Date}
\def\date@name{年度考核日期}
\fi

\ifxueweimaster
\def\makecover{
  \def\temp{English}
  \ifx\temp\preferlanguage
    % English
    \thispagestyle{empty}
    \zihao{-2}\vspace*{10mm}
		\renewcommand{\CJKglue}{\hskip 2pt plus 0.08\baselineskip}
    \centerline{\zihao{2}\textbf{Southern University of Science and Technology}}

		\vspace{10mm}
		\centerline{\zihao{2}\textbf{\report@name@en}}

		\renewcommand{\CJKglue}{\hskip 0pt plus 0.08\baselineskip}
\vspace{30pt}
\zihao{-2}
\begin{center}\textbf{Title:\@title}\end{center}
\vspace{30pt}
    \zihao{3}
    \hspace*{30pt}{\songti
	\renewcommand{\arraystretch}{1.3}
    \begin{tabular}{l@{}l}
    \textbf{Department}   & \rule[-4pt]{250pt}{1pt}\hspace{-352pt}\centerline{\textbf\@affil}\\
    \textbf{Discipline}     & \rule[-4pt]{250pt}{1pt}\hspace{-352pt}\centerline{\textbf\@subject}\\
    \textbf{Supervisor}     & \rule[-4pt]{250pt}{1pt}\hspace{-352pt}\centerline{\textbf\@supervisor}\\
    \@assosupervisor
	\@cosupervisor
    \textbf{Student Name}      & \rule[-4pt]{250pt}{1pt}\hspace{-352pt}\centerline{\textbf\@author}\\
    \textbf{Student Number}  & \rule[-4pt]{250pt}{1pt}\hspace{-352pt}\centerline{\textbf\@stuno}\\
    \textbf{\date@name@en} & \rule[-4pt]{250pt}{1pt}\hspace{-352pt}\centerline{\textbf\@date}\\
    \end{tabular}
		\renewcommand{\arraystretch}{1}}
	\vfill
    \centerline{\songti\textbf{Graduate School}}

  \else
    % Chinese
    \thispagestyle{empty}
    \zihao{-2}\vspace*{10mm}
		\renewcommand{\CJKglue}{\hskip 2pt plus 0.08\baselineskip}
    \centerline{\kaishu\textbf{南方科技大学}}

		\vspace{10mm}
		\centerline{\zihao{2}\songti\textbf{\report@name}}

		\renewcommand{\CJKglue}{\hskip 0pt plus 0.08\baselineskip}
\vspace{30pt}
\zihao{-2}
\begin{center}\songti\textbf{题~目:\@title}\end{center}
\vspace{30pt}
    \zihao{3}
    \hspace*{68pt}{\songti
	\renewcommand{\arraystretch}{1.3}
    \begin{tabular}{l@{}l}
    \textbf{院\hfill (系)}   & \rule[-4pt]{200pt}{1pt}\hspace{-326pt}\centerline{\textbf\@affil}\\
    \textbf{学\hfill 科}     & \rule[-4pt]{200pt}{1pt}\hspace{-326pt}\centerline{\textbf\@subject}\\
    \textbf{导\hfill 师}     & \rule[-4pt]{200pt}{1pt}\hspace{-326pt}\centerline{\textbf\@supervisor}\\
    \@assosupervisor
	\@cosupervisor
    \textbf{研\hfill 究\hfill 生}      & \rule[-4pt]{200pt}{1pt}\hspace{-326pt}\centerline{\textbf\@author}\\
    \textbf{学\hfill 号}  & \rule[-4pt]{200pt}{1pt}\hspace{-326pt}\centerline{\textbf\@stuno}\\
    \textbf{\date@name} & \rule[-4pt]{200pt}{1pt}\hspace{-326pt}\centerline{\textbf\@date}\\
    \end{tabular}
		\renewcommand{\arraystretch}{1}}
	\vfill
    \centerline{\songti\textbf{研究生院制}}
  \fi
}
\fi
\makeatother
